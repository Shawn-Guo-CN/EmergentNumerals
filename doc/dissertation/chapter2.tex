\chapter{Background}
\label{ch2:background}

There are 2 almost disjointly developed research topics that motivates this project, i.e. computer simulation methods in evolutionary linguistics and multi-agent games in grounded language learning. Thus, in the following 2 sections, we will introduce works which are highly related to our project from these 2 different areas.

\section{Computer Simulation Methods in Evolutionary Linguistics}
\label{sec2.1:evolang}

One important issue in the field of evolutionary linguistics is to use quantitative methods to overcome the time limit on unpreserved pre-historic linguistic behaviours \cite{lieberman2006toward, evans2009myth}. Since it was first introduced by \cite{hurford1989biological}, computer simulation method has attracted a rapidly growing attention, e.g. \cite{hurford1998approaches, knight2000evolutionary, briscoe2002book, christiansen2003language, bickerton2009biological, cangelosi2012simulating}. As introduced in Chapter \ref{ch1:intro}, one of our objectives is to facilitate the discovery and development of various kinds of natural language phenomena of computational agents, which shares a same objective and motivation of computer simulation methods in evolutionary linguistics.

To imply and verify a linguistic theory, there are 2 necessary component: i) environments, in which agents can execute actions and communicate with each other; ii) pre-defined elementary linguistic knowledge that can be manipulated and altered by agents. Further, we could categorise the environments into the following 3 different types according to their simulation objectives:
\begin{itemize}
  \item \textit{Iterated learning} introduced by \cite{kirby1999function} which aims at simulating cultural transitions from generation to generation.
  \item \textit{Language games} introduced by \cite{wittgenstein2009philosophical} which takes the emergent communication protocols in cooperation between individuals as a prototype of language.
  \item \textit{Genetic evolution} introduced by \cite{briscoe1998language} which aims at simulating evolution of languages as a kind of natural selection procedure\cite{darwin1859origin}.
\end{itemize}

With environments and pre-defined elementary linguistic knowledge, computational agents can then learn bi-directional meaning–utterance mapping functions\cite{gong2013computer}. With different kinds of resulting linguistic phenomena, this simulation procedure can be broadly categorised into 2 classes:
\begin{itemize}
  \item lexical models, e.g. \cite{steels2005emergence, baronchelli2006sharp, puglisi2008cultural}, whose main concern is whether a common lexicon can form during the communication in agent community;
  \item syntactic and grammatical models, e.g. \cite{kirby1999function, vogt2005acquisition}, in which agents mainly aim to map meanings (represented in various ways) to utterances (either structured or unstructured ).
\end{itemize}

However, no matter how these mapping functions are learnt, e.g. by neural network models \cite{munroe2002learning} or by mathematical equations \cite{minett2008modelling, ke2008language}, the most basic elements of linguistics, e.g. meanings to communicate about and a signalling channel to employ, are all pre-defined.

In contract, although we also follow the framework of language games and train agents in an iterated learning fashion, the basic linguistic elements in our project are not provided from the outset any more and computational agents can specify meanings of symbols/utterances by themselves.

\section{Multi-agent Games in Grounded Language Learning}
\label{sec2.2:gll}

Unlike how we human beings learn and understand language, the current DL-based NLP techniques learn semantics from only large-scaled static textual materials. Thus, grounded language learning argues that computational agents also need to learn and understand languages by interacting with environments and grounding language into their experience and perceptions. With the recent rapid development of deep reinforcement learning (DRL), it has been shown that computational agents can master a variety of complex cognitive activities, e.g. \cite{mnih2015human, silver2017mastering}]. Thus, several papers in grounded language learning apply DRL techniques to allow agents to learn or invent natural languages\footnote{Strictly speaking, ``invent natural language'' should be called as ``invent communication protocols sharing compositionality like natural languages''. However, as our project is to facilitate compositionality in multi-agent communication protocols, we thus call these emergent communication protocols a kind of ``language" invented by agents}, such as \cite{hermann2017grounded, mordatch2018emergence, havrylov2017emergence, hill2017understanding}.

To verify language abilities of computational agents, previous works in grounded language learning usually follow the framework of language games, of which are mainly variants of referential games introduced by \cite{lewis2008convention}, e.g. \cite{hermann2017grounded, havrylov2017emergence}. Also, some works are more motivated by the competition instead of cooperation such as \cite{cao2018emergent}.

From another perspective, based on the number of participated agents, we can broadly categorise language games in GLL into the following 2 types:

\begin{itemize}
  \item \textit{Single-agent games} usually need to be done by one agent and a human participator, in which the main concern is to explore how computational agents could learn the compositionality of semantics.
  \item \textit{Multi-agent games} are usually completed by an agent population, in which the main concern is to explore how various kinds of natural language phenomena emerge and evolve during communicating among agents.
\end{itemize}

However, like we mentioned in Chapter \ref{ch1:intro}, whichever kind of language game they follow in previous works of grounded language learning, the objects/attributes the symbols grounded to are all referential. We, on the other hand, aim to explore the feasibility of grounding symbols to non-referential objects (specifically, numeric concepts) during the game.