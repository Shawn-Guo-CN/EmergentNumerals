%%%%%%%%%%%%%%%%%%%%%%%%
% Sample use of the infthesis class to prepare an MSc thesis.
% This can be used as a template to produce your own thesis.
% Date: June 2019
%
%
% The first line specifies style options for taught MSc.
% You should add a final option specifying your degree.
% *Do not* change or add any other options.
%
% So, pick one of the following:
% \documentclass[msc,deptreport,adi]{infthesis}     % Adv Design Inf
% \documentclass[msc,deptreport,ai]{infthesis}      % AI
% \documentclass[msc,deptreport,cogsci]{infthesis}  % Cognitive Sci
% \documentclass[msc,deptreport,cs]{infthesis}      % Computer Sci
% \documentclass[msc,deptreport,cyber]{infthesis}   % Cyber Sec
% \documentclass[msc,deptreport,datasci]{infthesis} % Data Sci
% \documentclass[msc,deptreport,di]{infthesis}      % Design Inf
% \documentclass[msc,deptreport,inf]{infthesis}     % Informatics
%%%%%%%%%%%%%%%%%%%%%%%%

\documentclass[msc,deptreport]{infthesis} % Do not change except to add your degree (see above).
\usepackage{breakcites}

\begin{document}
\begin{preliminary}

\title{Emergence of Numerals in Multi-Agent Autonomous Communication System}

\author{Shangmin Guo}

\abstract{
  This project aims to propose a new computational simulation method for the emergence of numerals based on multi-agent autonomous communication system following deep reinforcement learning methodology. 
}

\maketitle

\section*{Acknowledgements}
Any acknowledgements go here. 

\tableofcontents
\end{preliminary}


\chapter{Introduction}
\label{ch1:intro}

Natural language processing (NLP) is an important and long-standing topic in artificial intelligence (AI), in which a core question is natural language understanding (NLU). With the rapid development of deep learning (DL), most current statae-of-the-art methods in NLP, e.g. \cite{socher2013recursive, word2vec2013, kim2014cnn}, are based on DL models trained on massive static textual corpora. From an information processing perspective, I illustrate the global view of NLP-oriented human-computer interaction system in Figure given as follow. As we can see in the diagram, the input of NLP systems are various kinds of textual materials generated by human beings to descibe their experiences or perceptions. Under such a perspective, symbols in natural languages are actually representations of features of the origianl experiences or perceptions, whereas most NLP systems directly take these symbols as original features.

%% FIGURE HERE!!!!

Therefore, grounded language learning (GLL) argues that models need a grounded environment to learn and understand language\cite{matuszek2018grounded}. However, natural languages of the time have been developed for at least tens of thousands of years\cite{berwick2016only} and already became very sophisticated. Thus, to verify that computational agents can truly understand and complete the tasks specified by natural languages, it is necessary to facillate them to develop various kinds of characteristics of natural language during autonomous communication.

\section{Evolutionary Linguistics}
\label{sec1.1:evo_lang}

The emergence and evolution of natural language have always been critical questions to 

Such question is a critical question in evolutionary linguistics.

However, previous works have to pre-define the basic elements of language.

\section{Deep Reinforcement Learning and Its Application in Grounded Language Learning}
\label{sec1.2:DRL_GLL}

With the recent development of DRL, we do not need to pre-define any linguistic element and thus can simulate the emergence of preliminary linguistic phenonmanon.

With these promising progresses in GLL and characteristics of numerals, this project proposes a new simulation methods of the emergence of numeral systems and also methods to analyse them.

%------------------------------------------------------------------------------------------%

\chapter{Background}
\label{ch2:background}

A dissertation usually contains several chapters.

\section{Computer Simulation Methods in Evolutionary Linguistics}
\label{sec2.1:evolang}

\section{Multi-agent Games in Grounded Language Learning}
\label{sec2.2:gll}



%------------------------------------------------------------------------------------------%

\chapter{Set Generation Game and Models}
\label{ch3:game_model}

\section{Game Description}
\label{sec3.1:game_description}

One hypothesis of our work is that, the linguistic hypotheses can by implied by game dynamics.

\subsection{Game Procedure}
\label{ssec3.1.1:game_procedure}

\subsection{Numerals in the Game}
\label{ssec3.1.2:numeral_in_game}

\section{Proposed Models}
\label{sec3.2:models}

\subsection{Set2Seq2Seq Models}
\label{ssec3.2.1:set2seq2seq}

\subsection{Numeral Iterated Learning}
\label{ssec3.2.2:3phase}

\subsection{Baseline Models}
\label{ssec3.2.3:baselines}

%------------------------------------------------------------------------------------------%

\chapter{Experiment Results and Analysis}
\label{ch4:results_analysis}

%------------------------------------------------------------------------------------------%

\chapter{Conclusions}
\label{ch5:conclusion}

\section{Final Reminder}

The body of your dissertation, before the references and any appendices,
\emph{must} finish by page~40. The introduction, after preliminary material,
should have started on page~1.

You may not change the dissertation format (e.g., reduce the font
size, change the margins, or reduce the line spacing from the default
1.5 spacing). Over length or incorrectly-formatted dissertations will
not be accepted and you would have to modify your dissertation and
resubmit.  You cannot assume we will check your submission before the
final deadline and if it requires resubmission after the deadline to
conform to the page and style requirements you will be subject to the
usual late penalties based on your final submission time.

\section{Further Discussion}
\label{sec5.1:discusstion}

%------------------------------------------------------------------------------------------%

\bibliographystyle{apalike}
\bibliography{main}

\chapter*{Appendices}

%% You can include appendices like this:
% \appendix
% 
% \chapter{First appendix}
% 
% \section{First section}
% 
% Markers do not have to consider appendices. Make sure that your contributions
% are made clear in the main body of the dissertation (within the page limit).

\end{document}
