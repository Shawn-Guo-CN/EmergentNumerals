%%%%%%%%%%%%%%%%%%%%%%%%
% Sample use of the infthesis class to prepare an MSc thesis.
% This can be used as a template to produce your own thesis.
% Date: June 2019
%
%
% The first line specifies style options for taught MSc.
% You should add a final option specifying your degree.
% *Do not* change or add any other options.
%
% So, pick one of the following:
% \documentclass[msc,deptreport,adi]{infthesis}     % Adv Design Inf
% \documentclass[msc,deptreport,ai]{infthesis}      % AI
% \documentclass[msc,deptreport,cogsci]{infthesis}  % Cognitive Sci
% \documentclass[msc,deptreport,cs]{infthesis}      % Computer Sci
% \documentclass[msc,deptreport,cyber]{infthesis}   % Cyber Sec
% \documentclass[msc,deptreport,datasci]{infthesis} % Data Sci
% \documentclass[msc,deptreport,di]{infthesis}      % Design Inf
% \documentclass[msc,deptreport,inf]{infthesis}     % Informatics
%%%%%%%%%%%%%%%%%%%%%%%%

\documentclass[msc,deptreport]{infthesis} % Do not change except to add your degree (see above).

\begin{document}
\begin{preliminary}

\title{Semantic Analysis of \\ Emergent Numerals in Multi-Agent Autonomous Communication system}

\author{Shangmin Guo}

\abstract{
  This skeleton demonstrates how to use the \texttt{infthesis} style
  for MSc dissertations in Artificial Intelligence, Cognitive Science,
  Computer Science, Data Science, and Informatics. It also emphasises
  the page limit, and that you must not deviate from the required
  style.  The file \texttt{skeleton.tex} generates this document and
  can be used as a starting point for your thesis. The abstract should
  summarise your report and fit in the space on the first page.
}

\maketitle

\section*{Acknowledgements}
Any acknowledgements go here. 

\tableofcontents
\end{preliminary}


\chapter{Introduction}
\label{ch1:intro}

NLU is a long-standing challenge.

Only massive textual materials are not sufficient for computers to understand our language.

We need import these ``biases" into computational agents.

It is necessary to facillate agants develop various kinds of characteristics of natural language during autonomous communication.

\section{Evolutionary Linguistics}
\label{sec1.1:evo_lang}

Such question is a critical question in evolutionary linguistics.

However, previous works have to pre-define the basic elements of language.

\section{Deep Reinforcement Learning and Its Application in Grounded Language Learning}
\label{sec1.2:DRL_GLL}

With the recent development of DRL, we do not need to pre-define any linguistic element and thus can simulate the emergence of preliminary linguistic phenonmanon.

With these promising progresses in GLL and characteristics of numerals, this project proposes a new simulation methods of the emergence of numeral systems and also methods to analyse them.

%------------------------------------------------------------------------------------------%

\chapter{Background}
\label{ch2:background}

A dissertation usually contains several chapters.

\section{Computer Simulation Methods in Evolutionary Linguistics}
\label{sec2.1:evolang}

\section{Multi-agent Games in Grounded Language Learning}
\label{sec2.2:gll}



%------------------------------------------------------------------------------------------%

\chapter{Set Generation Game and Models}
\label{ch3:game_model}

\section{Game Description}
\label{sec3.1:game_description}

One hypothesis of our work is that, the linguistic hypotheses can by implied by game dynamics.

\subsection{Game Procedure}
\label{ssec3.1.1:game_procedure}

\subsection{Language Game as a Markov Desicion Process}
\label{ssec3.1.2:game_mdp}

\subsection{Numerals in the Game}
\label{ssec3.1.3:numeral_in_game}

\section{Proposed Models}
\label{sec3.2:models}

\subsection{Set2Seq2Seq Models}
\label{ssec3.2.1:set2seq2seq}

\subsection{Baseline Models}
\label{ssec3.2.2:baselines}

%------------------------------------------------------------------------------------------%

\chapter{Experiment Results and Analysis}
\label{ch4:results_analysis}

%------------------------------------------------------------------------------------------%

\chapter{Conclusions}
\label{ch5:conclusion}

\section{Final Reminder}

The body of your dissertation, before the references and any appendices,
\emph{must} finish by page~40. The introduction, after preliminary material,
should have started on page~1.

You may not change the dissertation format (e.g., reduce the font
size, change the margins, or reduce the line spacing from the default
1.5 spacing). Over length or incorrectly-formatted dissertations will
not be accepted and you would have to modify your dissertation and
resubmit.  You cannot assume we will check your submission before the
final deadline and if it requires resubmission after the deadline to
conform to the page and style requirements you will be subject to the
usual late penalties based on your final submission time.

%------------------------------------------------------------------------------------------%

\bibliographystyle{apalike}
\bibliography{main}

\chapter*{Appendices}

%% You can include appendices like this:
% \appendix
% 
% \chapter{First appendix}
% 
% \section{First section}
% 
% Markers do not have to consider appendices. Make sure that your contributions
% are made clear in the main body of the dissertation (within the page limit).

\end{document}
