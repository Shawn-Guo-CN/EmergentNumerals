\documentclass[compress,mathserif,xcolor=dvipsnames,svgnames,aspectratio=43]{beamer}
%% $%!TEX program = xelatex
\usetheme{Algiers}

%%% My favorite packages

%\usepackage[default]{sourcesanspro}
% \usepackage[default]{raleway}

%\usepackage[T1]{fontenc}
%\usefonttheme[onlymath]{serif}

% \usepackage{fontspec}
% \usepackage{xunicode} %Unicode extras!
% \usepackage{xltxtra}  %Fixes
% \usepackage{Ubuntu}
% \usepackage[final,expansion=true,protrusion=true,spacing=true,kerning=true]{microtype} % better font output

%%% Really nice for tables
% \usepackage{booktabs}
% \newcommand{\ra}[1]{\renewcommand{\arraystretch}{#1}}
% \usepackage{float}

%%% Citations
% \usepackage{natbib}
% \renewcommand{\bibsection}{\subsubsection*{\bibname } } %prevents natbib from adding a section
\usepackage[english]{babel}
% \usepackage[utf8]{inputenc}
\usepackage{multirow}

% It puts a simplistic plain slide with the section name
% at the beginning of each section
\usepackage{nameref}
\makeatletter
\newcommand*{\currentname}{\@currentlabelname}
\makeatother
\AtBeginSection[]
{
\plaintitle{\currentname}
}

% Take this off when using the template
\usepackage{blindtext}
\usepackage{graphicx}

\setbeamertemplate{bibliography item}{\insertbiblabel}

%%%%%%%%%%%%%%%%%%%%%%%%%%%%%%%%%%%%%%%%%%%%%%%%%%%%%%%%%%
\begin{document}

\title{\huge{Current Progress in \\ Emergent Numeral Project}\\\medskip
    \large\textcolor{beautyblue}{} \medskip}
\author[{Shangmin (Shawn) Guo}]{\textcolor{grizoo}{Shangmin (Shawn) Guo}}
\date[someday\ldots]{\today}%{1\up{er} juillet 2010}
\institute{School of Informatics, The University of Edinburgh}

\begin{frame}[plain, noframenumbering]
\titlepage
\end{frame}

%% the content page
\begin{frame}[plain, noframenumbering]
\frametitle{Content}
\tableofcontents
\end{frame}

%% the first section
\section{Task Description \& Models}

\begin{frame}[c]
  \frametitle{Task}
  % Characters
\end{frame}

\begin{frame}[c]
  \frametitle{Non-communication Models}

\end{frame}

\begin{frame}[c]
  \frametitle{Communication Model}
  % 2 different methods
\end{frame}

%=============================================================================================%
\section{Experiments Results \& Preliminary Discussion}

\begin{frame}[c]
  \frametitle{Model Performances}
  
\end{frame}

\begin{frame}[c]
  \frametitle{Distribution of Messages}
  
\end{frame}

\begin{frame}[c]
  \frametitle{Limitation of Topological Similarity}
  
\end{frame}

\begin{frame}[c]
  \frametitle{Some Important Properties}
  \begin{enumerate}
    \item \textbf{Orthogonality \& Mutual Exclusivity}: 
    \item \textbf{Monotony \& Topological Similarity}: 
  \end{enumerate}
\end{frame}


%=============================================================================================%
\section{MISC}

\begin{frame}[c]
  \frametitle{Explaination of Limiting $|V|$ and $L$}
  
\end{frame}

\begin{frame}[c]
  \frametitle{Some Related Works}
  \cite{li2019ease}
\end{frame}


% A trick for having back slides while ending the slide counter
% on the conclusion slide
\appendix
\newcounter{finalframe}
\setcounter{finalframe}{\value{framenumber}}

% Geeky ending
\begin{frame}[c,plain,fragile] %fragile
%\begin{python}
%feedback = raw_input( 'Questions ?' )
%if '?' in feedback:
%    if have_answer():
%        give_answer()
%    else:
%        pretend_the_question_is_ill_posed()
%else:
%    print 'Thanks, let us go get MIX now.'
%\end{python}

% Or simply...
\begin{center}
\huge{Thank you}
\end{center}
\end{frame}

% References
\begin{frame}[allowframebreaks, plain]
\tiny{
   \bibliographystyle{amsalpha}
   \bibliography{slides}
}
\end{frame}

\setcounter{framenumber}{\value{finalframe}}
\end{document}
